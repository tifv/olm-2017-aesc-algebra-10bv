% $date: 2016-09-09

\section*{Системы линейных уравнений --- 1. Метод Гаусса}

% $build$matter[print]: [[.], [.]]

\begin{problems}

\itemx{$^\circ$}
Решите системы:\begin{tabbing}
\begin{subproblemeq}
\left\{\begin{aligned} &
      x - 3 y + 2 z -   t =  3
\\ &
    2 x + 4 y - 3 z +   t =  5
\\ &
    4 x - 2 y +   z +   t =  3
\\ &
    3 x +   y +   z - 2 t = 10
\end{aligned}\right.
\end{subproblemeq}
\qquad\=
\begin{subproblemeq*}
\left\{\begin{aligned} &
    x + 2 y + 3 z -   t =  0
\\ &
    x -   y +   z + 2 t =  4
\\ &
    x + 5 y + 5 z - 4 t = -4
\\ &
    x + 8 y + 7 z - 7 t = -8
\end{aligned}\right.
\end{subproblemeq*}
\\
\begin{subproblemeq}
\left\{\begin{aligned} &
      x + 2 y + 3 z =  2
\\ &
      x -   y +   z =  0
\\ &
      x + 3 y -   z = -2
\\ &
    3 x + 4 y + 3 z =  0
\end{aligned}\right.
\end{subproblemeq}
\qquad\>
\begin{subproblemeq*}
\left\{\begin{aligned} &
    x + 2 y + 3 z -   t =  0
\\ &
    x -   y +   z + 2 t =  4
\\ &
    x + 5 y + 5 z - 4 t = -4
\\ &
    x + 8 y + 7 z - 7 t =  6
\end{aligned}\right.
\end{subproblemeq*}
\end{tabbing}

\end{problems}

\emph{Однородная система}~--- это линейная система, в которой все правые части
равны нулю.

\begin{problems}

\item
Пусть количество неизвестных линейной системы совпадает с количеством
уравнений.
Докажите, что линейная система в этом случае является совместной и определенной
тогда и только тогда, когда соответствующая ей однородная система имеет только
нулевое решение.

\item
Система линейных уравнений с действительными коэффициентами имеет два различных
решения.
Постройте по ним бесконечную серию различных решений.

\item
Найдите квадратный трехчлен $f(x)$, зная, что
\[
    f(1) = -1
,\quad
    f(-1) = 9
,\quad
    f(2) = -3
.\]

\item
Имеется система линейных уравнений
\[ \def\star{\mathord{\ast}}
\left\{\begin{aligned} &
    \star x + \star y + \star z = 0
\\ &
    \star x + \star y + \star z = 0
\\ &
    \star x + \star y + \star z = 0
\end{aligned}\right.\]
Два человека вписывают по очереди вместо звездочек действительные числа.
Докажите, что начинающий всегда может добиться того, чтобы система имела
ненулевое решение.

\item
Найдите все решения системы бесконечного числа уравнений
\[
    x \left( 1 - \frac{1}{2^n} \right)
    +
    y \left( 1 - \frac{1}{2^{n+1}} \right)
    +
    z \left( 1 - \frac{1}{2^{n+2}} \right)
=
    0
,\quad
    n = 1, 2, 3, 4, \ldots
\]

\end{problems}

